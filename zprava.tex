\documentclass[12pt]{article}
\usepackage[czech]{babel}
\usepackage{epsf,epic,eepic,eepicemu, graphicx}
%\documentstyle[epsf,epic,eepic,eepicemu]{article}

\usepackage[T1]{fontenc}
\usepackage[utf8]{inputenc}
\usepackage{cmap}
\usepackage{mathpazo}

\begin{document}
%\oddsidemargin=-5mm \evensidemargin=-5mm \marginparwidth=.08in
%\marginparsep=.01in \marginparpush=5pt \topmargin=-15mm
%\headheight=12pt \headsep=25pt \footheight=12pt \footskip=30pt
%\textheight=25cm \textwidth=17cm \columnsep=2mm \columnseprule=1pt
%\parindent=15pt\parskip=2pt

\begin{center}
\bf České vysoké učení technické v Praze\\[2mm]
	Fakulta informačních technologií\\[2mm]
	Thákurova 9, 160 00, Praha 6\\[15mm]
	\textbf{Semestralní projekt MI-PAR 2011/2012:\\
    Paralelní algoritmus \\
    zobecněné Hanojské věže}\\[5mm]
       Jaroslav Hejral\\
       Jan Langer\\[2mm]
\today\\[12mm]
\end{center}

\section{Definice problému a popis sekvenčního algoritmu}
Algoritmus úlohy řeší problém sestavení postupu tahů k vyřešení zoběcněných
Hanojských věží pro počet věží větší než 3. Hanojská věž o výšce \verb|k| je věž
\verb|k| různých žetonů, které jsou uspořádány od nejmenšího k největšímu a
rozdíly ve velikostech sousedních žetonů jsou vždy 1. Oproti tomu neúplná
hanojská věž o výšce \verb|k| je věž \verb|k| různých žetonů, které jsou
uspořádány od nejmenších k největším a rozdíly ve velikostech alespoň 1 dvojice
sousedních žetonů je alespoň 2. Jeden tah je přesun disku z jedné tyče na jinou
s tím, že cílová tyč je buď prázdná nebo je na jejím vrcholu disk s větším
průměrem. Úkolem algoritmu je sestavit na cílové tyčce úplnou hanojskou věž
pomocí minimálního počtu tahů.

\subsection{Vstupní data}
\begin{itemize}
\item n = přirozené číslo představující celkový počet žetonů, n >=
16. Žeton i, i=1,..,n, má průměr i
\item s = přirozené číslo představující počet tyček, n/4 >=s > 3
\item r = číslo cílové tyčky, 1 <= r <= s
\item V[1,..,s] = množina neúplných hanoiských věží.
\end{itemize}

Argument V se zadává pomocí skóre jednotlivých věží. Skóre tyče je desítková reprezentace binárního kódu představujícího velikosti disků umístěných na konkrétní tyči. Vypočte se pomocí vzorce:
\begin{center}
$\displaystyle\sum\limits_{i=0}^n {2^{d(i)-1}}$
\end{center}
 kde n=počtu disků na tyči a d(i)=velikosti disku. Součet skóre všech tyčí musí být rovno $2^{n-1}$.

Příklad volání programu s parametry:
\texttt{./mi-par 5 4 4 1:30:0:0}

\begin{figure}[h]
\begin{center}
\includegraphics[width=140mm]{5-4-4.png}
\caption{Počáteční konfigurce pro zadání \texttt{./mi-par 5 4 4 1:30:0:0}}
\end{center}
\end{figure}



\subsection{Výstup algoritmu}
Výstupem algortmu je výpis počtu tahů a posloupnost tahů ve formátu disk, původní tyč -> cílová tyč.
Přiklad výstupu pro ukázkové zadání:

\begin{verbatim}
-------------
I have solution!
Number of moves: 12
2: 2 > 4;
3: 2 > 3;
2: 4 > 3;
1: 1 > 3;
4: 2 > 1;
5: 2 > 4;
1: 3 > 2;
4: 1 > 4;
2: 3 > 1;
3: 3 > 4;
2: 1 > 4;
1: 2 > 4;
\end{verbatim}
\subsection{Popis sekvenčního algoritmu}
Algoritmus je typu BB-DFS, tedy prohledávání do hloubky s návratem a mezí.
Řešení algoritmu vždy existuje. 

V každém cyklu sekvenčního algoritmu je odebrán prvek z vrcholu zásobníku.
Odebrané částečné řešení je zkontrolováno zda se nejedná o úplné řešení. Pokud
ano, je řešení uloženo smyčka programu se spustí znovu se sníženou hodnotou
horní meze na hodnotu o jedna menší než je právě nalezené řešení. V případě, že
se nejedná o konečné řešení, zkontroluje se dosažení horní meze. V případě, že
horní mez nebyla dosažena, vygenerují se všechny nová částečná řešení o jeden
krok hlubší než původní částečné řešení a vloží se do zásobníku. Hlavní smyčka
programu běží dokud není prázdný zásobník. Na konci běhu program vypíše nalezené
řešení a počet tahů.

\subsubsection{Optimalizace ořezávání horní mezí}
Těsná horní mez řešení se vypoítá dle vzorce $(2^{n/(s-2)-1})(2s-5)$. V případě
že je nalezeno řešení je horní mez snížena na jeho velikost mínus jedna (hledáme
o jedno menší řešení). Pokud zanoření ve stromu řešení překročí aktuálně platnou
horní mez dojde k oříznutí této neperspektivní větve a provede se návrat. 

Ořezávání se nám ještě podařilo optimalizovat, a to tak, že návrat se provede už
v případě, že aktuálně zbývající počet kroků do horní meze již nestačí pro
přesunutí všech zbývajících disků na cílovou tyč. Touto optimalizací můžeme
dosáhnout oříznutí neperspektivního postupu až o (horní mez - počet disků)
dříve. Její významnost se zvyšuje s přibývajícím množstvím disků.

Další podstatnou optimalizaci jsme dosáhli zamezením zpětných smyček kdy se
žeton zaczklí mezi dvěma tyčemi (tam a zpátky). Původně jsme uvažovali zamezení
i vícekrokových smyček, nicméně režije na jejich kontrolu vedla naopak ke
snížení efektivity.

\section{Popis paralelního algoritmu a jeho implementace v MPI}
Algoritmus je typu \verb|G-PBB-DFS-D|. Oproti zádání jsme upravili způsob
šíření nejlepšího známého řešení, které jsme se rozhodli sdílet globálně. Vedla
nás k tomu především vidina značné uspory ve výpočtech. Procesor, který nalezne
řesení (které z podstaty úlohy vždy existuje) sdílí jeho délku s ostatními
procesory. To nám u nich umožní rychleji zaříznout částečná řešení, která již
nemají šanci na to býti lepšími.

\subsection{Pracovní smyčka}
Hlavní smyčka programu, která slouží pro řešení problému je prováděna dokud má
daný procesor něco na práci (v zásobníku disponuje nějakým částečným řešením,
které ještě neprozkoumal). Kromě standardního výpočtu řešení, který odpovídá
sekvenčnímu algoritmu obsahuje neblokovanou detekci příchodu zpráv. V této
části kódu můžou přijít následující zprávy.

\begin{description}
  \item [TAG\_NEED\_MORE\_WORK] Některý procesor žádá práci. Pokud danému 
procesoru ještě zbývá nějaká práce v zásobníku, tak žadateli pošle práci ze 
dna zásobníku. Distribucí práce ze spodku zásobníku jsme dosáhli značného 
zrychlení neboť zde máme jistotu, že předané částečné řešení, má největší
potenciál další řešitelnosti.
  \item [TAG\_SOLUTION\_NEW\_LIMIT] Tato zprávu rozesílá pouze hlavní
procesor(P0). Každý procesor, který ji přijme si upraví vlastní limit určující
novou horní mez pro hledání řešení. 
  \item [TAG\_FINALIZE] Tuto zprávu rozesílá pouze předchůdce a nejedná
se o nic jiného než o peška. Při přijmu se provede logický součin peška s
vlastním indikátorem práce. Agresivní je v tomto přípádě 0, která reprezentuje
existenci práce. V rámci hlavní smyčky je krajně nepravděpodobné, že by procesor
dále poslal že nemá práci. Jediný případ, kdy by to mohlo nastat je, když by
procesor těsně před přijmutím peška odeslal poslední částečné řešení žadateli o
práci.
\end{description}
V neposlední řadě se v rámci pracovní smyčky kontroluje dosažení řešení. V
případě, že procesor najde řešení splňující kriteria, pošle toto řešení hlavnímu
procesoru(P0). Posílá mu počet kroků vedoucí k řešení a řešení jako takové.
Hlavní procesor na to reaguje tak, že napřed porovná počet kroků s jeho
aktuálním limitem a pokud je dané řešení kratší, uloží si jej a pošle všem
procesorům zprávu \verb|TAG_SOLUTION_NEW_LIMIT| s novým limitem.

\subsection{Čekací smyčka}
Do čekací smyčky se procesor dostane v případě, že mu dojde práce. Postupně
neblokovaně žádá o práci a následně čeká na přijetí jedné z následujících zpráv.
\begin{description}
  \item [TAG\_SIZE\_OF\_NEW\_WORK] Jedná se o první ze dvou zpráv s novou prací.
Principielně procesor požádal o práci a také ji dostal. Získanou práci si vloží
do zásobníku, změní svůj stav na to, že má práci a opustí smyčku hledání práce.
  \item [TAG\_FINALIZE] Jedná o peška. Pokud mi je poslán v tuto chvíli vím, že
nejsem jeho autorem, a proto ho pouze logicky přenásobím se svým stavem a sice
nemám práci (logická 1). Opět platí agresivita nuly (je práce).  
  \item [TAG\_NO\_MORE\_WORK] Tuto zprávu očekávám pouze od toho procesoru,
kterého jsem žádal o práci. Indikuje danému procesoru fakt, že ani dotazovaný
nemá nic na práci, a proto se musí poohlédnout jinde.
  \item [TAG\_NEED\_MORE\_WORK] Někdo od daného procesoru žádá práci. Bohužel
ji nemá a proto rovnou odpoví zprávou \verb|TAG_NO_MORE_WORK|.
   \item [TAG\_TERMINATE] Někdo rozhodl, že je na čase ukončit výpočet. Po
přijmu této zprávy napřed daný procesor přepošle asynchronně stejnou zprávu
svému následovníkovi a poté se ukončí. Hlavní procesor (P0) před ukončením
ještě vypíše na stardandní výstup řešení.
\end{description}
V případě, že opouští procesor část čekací smyčky ve které si řešil hledání
prace a tento proces byl úspěšný, vrátí se do pracovní smyčky. V případě
neúspěchu, začne šířit peška. Blokovaně čeká na jeho návrat od předchudce. Pokud
se pešek vrátí s tím, že někde ještě práce je, vrátí se do části vyhledávání
práce. Pokud se však pešek vratí s tím, že už nikde práce není, začne
procesor šířit zprávu  \verb|TAG_TERMINATE| a ukončí se.
\subsection{Hledání dárce}
Pro hledání dárce jsme se rozhodli použít pseudo-náhodný přistup. Procesor,
kterému dojde práce si náhodně určí první procesor, kterého požádá o práci.
Pokud od něj práci nedostane tak požádá jeho následovníka. Toto se opakuje
dokud nevyčerpá všechny možnosti.

\subsection{Ukončení výpočtu}
Pro ukončení výpočtu používáme stardandní kolečko s peškem. Peška začne posílat
procesor, který nemá prací i poté, co o ní požádal postupně všechny ostatní procesy. 

Každý kdo peška přijme ho přebarví podle toho, jestli práci má nebo ne. Případ,
kdy přijde pešek v momentě kdy procesor poslal práci do části, kde již pešek
byl, máme ošetřen tím, že v tu chvíli se daný procesor tváří jako by sám práci
měl, i když to nemusí být nezbytně pravda. 

Když se pešek vrátí k procesoru který ho vyslal, a je ve stavu \uv{není práce}
(logická 1), informuje svého následníka o ukončení výpočtu. Každý procesor,
který zprávu o ukončení obdrží, ji přepošle svému následovníkovi a ukončí se.
Jedinou vyjímkou v pravidlu je hlavní procesor (P0), který před svým ukončením
vypíše na výstup nalezené řešení.

\subsection{Sémantika příkazu}
Oproti sekvenčnímu řešení nedoznala sémantika příkazu žádných změn. 
\begin{verbatim}
  ./mi-par 5 4 4 1:30:0:0                 
\end{verbatim}

\section{Naměřené výsledky a vyhodnocení}

Výslednou aplikaci jsme otestovali na výpočetním stroji STAR pro tři různě velké
instance problému. Každé zadání bylo změřeno pro běh na 1, 2, 4, 8, 16 a 24
procesorech a pro komunikační sítě InfiniBand a Ethernet. Měření na svazku STAR
probíhalo vždy na uzlech s 12 jádry (parametr \textit{12c} při spuštění).

\begin{itemize}
\item Zadání č. 1: \texttt{16 6 6 72:32:3:0:20:65408}
\item Zadání č. 2: \texttt{16 7 7 72:32:3:0:20:0:65408}
\item Zadání č. 3: \texttt{16 9 9 72:32:3:128:20:0:0:0:65280}



\end{itemize}

\begin{figure}[h]
\begin{center}
\includegraphics[width=140mm]{16-6-6.png}
\caption{Zadání č. 1: 16 disků, 6 tyčí, cílová tyč č. 6}
\end{center}
\end{figure}

\begin{figure}[h]
\begin{center}
\includegraphics[width=140mm]{16-7-7.png}
\caption{Zadání č. 2: 16 disků, 7 tyčí, cílová tyč č. 7}
\end{center}
\end{figure}

\begin{figure}[h]
\begin{center}
\includegraphics[width=140mm]{16-9-9.png}
\caption{Zadání č. 3: 16 disků, 9 tyčí, cílová tyč č. 9}
\end{center}
\end{figure}


\subsection{Naměřené výsledky pro síť InfiniBand 10Gbps}
\begin{center}
\begin{tabular}{|c|c|c|c|}
\hline 
 & \multicolumn{3}{c|}{čas [s] (zaokrouhleno na 1 desetiné místo)} \\ 
\hline 
počet procesů & Zadání č.1 & Zadání č.2 & Zadání č.3 \\ 
\hline 
\hline 
1 & 2387,5 & 379,2 & 1251,2 \\ 
\hline 
2 & 1095,8 & 202,9 & 793,0 \\ 
\hline 
4 & 289,1 & 36,1 & 415,5 \\ 
\hline 
8 & 177,8 & 35,5 & 228,1 \\ 
\hline 
16 & 176,6 & 35,7 & 224,2 \\ 
\hline 
24 & 175,4 & 34,9 & 220,9 \\ 
\hline 
\end{tabular} 
\end{center}


\begin{figure}[h]
\begin{center}
\includegraphics[width=100mm]{cpu_time_inifiniband.png}
\caption{Rychlost výpočtu v závislosti na počtu procesů (InfiniBand)}
\label{fig:ct_inifini}
\end{center}
\end{figure}

\begin{figure}[h]
\begin{center}
\includegraphics[width=100mm]{speedup_infini.png}
\caption{Zrychlení výpočtu v závisloti na počtu procesů (InfiniBand)}
\label{fig:speedup_inifini}
\end{center}
\end{figure}

\subsection{Naměřené výsledky pro síť Ethernet 1Gbps}

\begin{center}
\begin{tabular}{|c|c|c|c|}
\hline 
 & \multicolumn{3}{c|}{čas [s] (zaokrouhleno na 1 desetiné místo)} \\ 
\hline 
počet procesů & Zadání č.1 & Zadání č.2 & Zadání č.3 \\ 
\hline 
\hline 
1 & 2387,5 & 379,2 & 1251,2 \\ 
\hline 
2 & 1151,3 & 205,5 & 798,4 \\ 
\hline 
4 & 294,2 & 41,2 & 435,5 \\ 
\hline 
8 & 181,4 & 37,0 & 235,5 \\ 
\hline 
16 & 181,9 & 36,1 & 227,1 \\ 
\hline 
24 & 181,7 & 35,4 & 221,1 \\ 
\hline 
\end{tabular} 
\end{center}

\begin{figure}[h]
\begin{center}
\includegraphics[width=100mm]{cpu_time_ethernet.png}
\caption{Rychlost výpočtu v závislosti na počtu procesů (Ethernet)}
\label{fig:ct_inifini}
\end{center}
\end{figure}


\subsection{Porovnání měření na Infinibandu a Ethernetu}
Z porovnání hodnot naměřených na síti Infiniband a Ethernet je vidět, že rozdíli
dosahují maximálně desítek sekund pro větší instance na menším počtu procesorů.
Částečný vliv na hodnoty má samozřejmě také momentální zatížení svazku STAR a
jeho komunikačních sítí v době měření. V procentuálních hodnotách má nějvětší
rozdíl hodnotu 5\% pro zadání č.1 na 2 jádrech, pro menší instance a také větší
počet procesorů jsou ale rozdíly menší a tak můžeme říci, že rozdílná rychlost
přenosu mezi uzly nemá na celkovou dobu výpočtu zásadní vliv.

\section{Závěr}

Úspěšně se nám podařilo implementovat sekvenční i paralelní algoritmu pro
řešení zadaného problému. Provedli jsme měření času potřebného pro výpočet úloh
netriviální složitosti sekvenčně a paralelně na 1-24 jádrech procesoru. Výsledky
měření ukazují, že algoritmus vykazuje na více procesorech superlineární
zrychlení, a to především proto, že současným  výpočtem ve více větvích stromu
dokážeme dříve nalézt \uv{neideální} řešení a tedy snižovat horní mez, která je
globálně sdílena. 

Paradoxně nejvíce problému při implementaci nám dělala mala znalost použitého
programovcího jazyka. Především ladění chyb způsobených špatnou
alokací/dealokací pamětí nás vždy značně brzdilo.

Pokud jde o MPI jako takové, narazili jsme pouze na jeden problém. Špatně jsme
pochopili princip určování velikosti bufferu při přijmu zprávy. Měli jsme za to,
že je nutné použít zcela přesně velikost celé zprávy, a proto jsme každou zprávu
s novou prací nebo řešením (jakožto dinamicky dlouhé zprávy) rozdělili na dvě
části. V první zprávě se pošle velikost věty a teprve poté věta jako taková.
Bohužel jsme náš omyl zjistili až příliž pozdě na to to změnit. Proto jsou námi
naměřené výsledky paralelního algoritmu o něco horší, než by mohli být.

Semestrální práce pro nás byla přínosem, hlavně co se týká seznámení se s MPI a
obecně komunikací v paralelním prostředí. 

\end{document}
