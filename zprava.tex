\documentclass[12pt]{article}
\usepackage[czech]{babel}
\usepackage{epsf,epic,eepic,eepicemu, graphicx}
%\documentstyle[epsf,epic,eepic,eepicemu]{article}

\usepackage[T1]{fontenc}
\usepackage[utf8]{inputenc}
\usepackage{cmap}
\usepackage{mathpazo}

\begin{document}
%\oddsidemargin=-5mm \evensidemargin=-5mm \marginparwidth=.08in
%\marginparsep=.01in \marginparpush=5pt \topmargin=-15mm
%\headheight=12pt \headsep=25pt \footheight=12pt \footskip=30pt
%\textheight=25cm \textwidth=17cm \columnsep=2mm \columnseprule=1pt
%\parindent=15pt\parskip=2pt

\begin{center}
\bf České vysoké učení technické v Praze\\[2mm]
	Fakulta informačních technologií\\[2mm]
	Thákurova 9, 160 00, Praha 6\\[15mm]
	\textbf{Semestralní projekt MI-PAR 2011/2012:\\
    Paralelní algoritmus \\
    zobecněné Hanojské věže}\\[5mm]
       Jaroslav Hejral\\
       Jan Langer\\[2mm]
\today\\[12mm]
\end{center}

\section{Definice problému a popis sekvenčního algoritmu}
Algoritmus úlohy řeší problém sestavení postupu tahů k vyřešení zoběcněných Hanojských věží pro počet věží větší než 3. Hanojská věž o výšce k je věž k různých žetonů, které jsou uspořádány od nejmenšího k největšímu a rozdíly ve velikostech sousedních žetonů jsou vždy 1. Neuplná hanojská věž o výšce k je věž k různých žetonů, které jsou uspořádány od nejmenších k největším a rozdíly ve velikostech alespoň 1 dvojice sousedních žetonů je alespoň 2. Jeden tah je přesun disku z jedné tyče na jinou s tím, že cílová tyč je buď prázdná nebo je na jejím vrcholu disk s větším průměrem. Úkolem algoritmu je sestavit na cílové tyčce úplnou hanojskou věž s minimálním počtem tahů.

\subsection{Vstupní data}
\begin{itemize}
\item n = přirozené číslo představující celkový počet žetonů, n >= 16. Žeton i, i=1,..,n, má průměr i
\item s = přirozené číslo představující počet tyček, n/4 >=s > 3
\item r = číslo cílové tyčky, 1 <= r <= s
\item V[1,..,s] = množina neúplných hanoiských věží.
\end{itemize}

Argument V se zadává pomocí skóre jednotlivých věží. Skóre tyče je desítková reprezentace binárního kódu představujícího velikosti disků umístěných na konkrétní tyči. Vypočte se pomocí vzorce:
\begin{center}
$\displaystyle\sum\limits_{i=0}^n {2^{d(i)-1}}$
\end{center}
 kde n=počtu disků na tyči a d(i)=velikosti disku. Součet skóre všech tyčí musí být rovno $2^{n-1}$.

Příklad volání programu s parametry:
\texttt{./mi-par 5 4 4 1:30:0:0}

\begin{figure}[h]
\begin{center}
\includegraphics[width=140mm]{5-4-4.png}
\caption{Počáteční konfigurce pro zadání \texttt{./mi-par 5 4 4 1:30:0:0}}
\end{center}
\end{figure}



\subsection{Výstup algoritmu}
Výstupem algortmu je výpis počtu tahů a posloupnost tahů ve formátu disk, původní tyč -> cílová tyč.
Přiklad výstupu pro ukázkové zadání:

\begin{verbatim}
-------------
I have solution!
Number of moves: 12
2: 2 > 4;
3: 2 > 3;
2: 4 > 3;
1: 1 > 3;
4: 2 > 1;
5: 2 > 4;
1: 3 > 2;
4: 1 > 4;
2: 3 > 1;
3: 3 > 4;
2: 1 > 4;
1: 2 > 4;
\end{verbatim}
\subsection{Popis sekvenčního algoritmu}
Algoritmus je typu BB-DFS, tedy prohledávání do hloubky s návratem a mezí. Řešení algoritmu vždy existuje. 

V každém cyklu sekvenčního algoritmu je odebrán prvek na vrcholu zásobníku. Je zkontrolováno, zda-li jsme nedošli k řešení, pokud ano je řešení uloženo, sníží se horní mez a smyčka se spustí znovu. Pokud ne, zkontroluje se dosažení meze a zda-li existují možnosti přesunu disků do dalšího kroku. V případě je možné nějaké disky přesunout, vygenerují se  a vloží do zásobníku. Smyčka běží dokud není prázdný zásobník. Na konci běhu program vypíše nalezené řešení a počet tahů.

\subsubsection{Optimalizace ořezávání horní mezí}
Těsná horní mez řešení se vypočte podle vzorce $(2^{n/(s-2)-1})(2s-5)$. V případě že je nalezeno řešení je horní mez snížena na jeho velikost mínus jedna (hledáme o jedno menší řešení). Pokud zanoření ve stromu řešení překročí aktuálně platnou horní mez dojde k oříznutí této neperspektivní větve a provede se návrat. 

Ořezávání se nám ještě podařilo optimalizovat, a to tak, že návrat se provede už v případě, že aktuálně zbývající počet kroků do horní meze již nestačí pro přesunutí všech zbývajících disků na cílovou tyč. Touto optimalizací můžeme dosáhnout oříznutí neperspektivního postupu až o (horní mez-počet disků) dříve. Její významnost se zvyšuje s přibývajícím množstvím disků.



//TODO: vysvětlit proč jsme se vykašlali na dolní mez

\section{Popis paralelního algoritmu a jeho implementace v MPI}

//TODO: zmínit změnu na G-PBB-DFS-D (globální paralelní BB-DFS s DDE) - nebo dokonce G-PBB-DFS-V (always exhaustive)?




Popište paralelní algoritmus, opět vyjděte ze zadání a přesně
vymezte odchylky, zvláště u algoritmu pro vyvažování zátěže, hledání
dárce, ci ukončení výpočtu.  Popište a vysvětlete strukturu
celkového paralelního algoritmu na úrovni procesů v MPI a strukturu
kódu jednotlivých procesů. Např. jak je naimplementována smyčka pro
činnost procesů v aktivním stavu i v stavu nečinnosti. Jaké jste
zvolili konstanty a parametry pro škálování algoritmu. Struktura a
sémantika příkazové řádky pro spouštění programu.

\section{Naměřené výsledky a vyhodnocení}

Výslednou aplikaci jsme otestovali na výpočetním stroji STAR pro tři různě velké instance problému. Každé zadání bylo změřeno pro běh na 1, 2, 4, 8, 16 a 24 procesorech a pro komunikační sítě InfiniBand a Ethernet. Měření na svazku STAR probíhalo vždy na uzlech s 12 jádry (parametr \textit{12c} při spuštění).

\begin{itemize}
\item Zadání č. 1: \texttt{16 6 6 72:32:3:0:20:65408}
\item Zadání č. 2: \texttt{16 7 7 72:32:3:0:20:0:65408}
\item Zadání č. 3: \texttt{16 9 9 72:32:3:128:20:0:0:0:65280}



\end{itemize}

\begin{figure}[h]
\begin{center}
\includegraphics[width=140mm]{16-6-6.png}
\caption{Zadání č. 1: 16 disků, 6 tyčí, cílová tyč č. 6}
\end{center}
\end{figure}

\begin{figure}[h]
\begin{center}
\includegraphics[width=140mm]{16-7-7.png}
\caption{Zadání č. 2: 16 disků, 7 tyčí, cílová tyč č. 7}
\end{center}
\end{figure}

\begin{figure}[h]
\begin{center}
\includegraphics[width=140mm]{16-9-9.png}
\caption{Zadání č. 3: 16 disků, 9 tyčí, cílová tyč č. 9}
\end{center}
\end{figure}


\subsection{Naměřené výsledky pro síť InfiniBand 10Gbps}
\begin{center}
\begin{tabular}{|c|c|c|c|}
\hline 
 & \multicolumn{3}{c|}{čas [s] (zaokrouhleno na 1 desetiné místo)} \\ 
\hline 
počet procesů & Zadání č.1 & Zadání č.2 & Zadání č.3 \\ 
\hline 
\hline 
1 & 2387,5 & 379,2 & 1251,2 \\ 
\hline 
2 & 1095,8 & 202,9 & 793,0 \\ 
\hline 
4 & 289,1 & 36,1 & 415,5 \\ 
\hline 
8 & 177,8 & 35,5 & 228,1 \\ 
\hline 
16 & 176,6 & 35,7 & 224,2 \\ 
\hline 
24 & 175,4 & 34,9 & 220,9 \\ 
\hline 
\end{tabular} 
\end{center}


\begin{figure}[h]
\begin{center}
\includegraphics[width=100mm]{cpu_time_inifiniband.png}
\caption{Rychlost výpočtu v závislosti na počtu procesů (InfiniBand)}
\label{fig:ct_inifini}
\end{center}
\end{figure}

\begin{figure}[h]
\begin{center}
\includegraphics[width=100mm]{speedup_infini.png}
\caption{Zrychlení výpočtu v závisloti na počtu procesů (InfiniBand)}
\label{fig:speedup_inifini}
\end{center}
\end{figure}

\subsection{Naměřené výsledky pro síť Ethernet 1Gbps}

\begin{center}
\begin{tabular}{|c|c|c|c|}
\hline 
 & \multicolumn{3}{c|}{čas [s] (zaokrouhleno na 1 desetiné místo)} \\ 
\hline 
počet procesů & Zadání č.1 & Zadání č.2 & Zadání č.3 \\ 
\hline 
\hline 
1 & 2387,5 & 379,2 & 1251,2 \\ 
\hline 
2 & 1151,3 & 205,5 & 798,4 \\ 
\hline 
4 & 294,2 & 41,2 & 435,5 \\ 
\hline 
8 & 181,4 & 37,0 & 235,5 \\ 
\hline 
16 & 181,9 & 36,1 & 227,1 \\ 
\hline 
24 & 181,7 & 35,4 & 221,1 \\ 
\hline 
\end{tabular} 
\end{center}

\begin{figure}[h]
\begin{center}
\includegraphics[width=100mm]{cpu_time_ethernet.png}
\caption{Rychlost výpočtu v závislosti na počtu procesů (Ethernet)}
\label{fig:ct_inifini}
\end{center}
\end{figure}


\subsection{Porovnání měření na Infinibandu a Ethernetu}
Z porovnání hodnot naměřených na síti Infiniband a Ethernet je vidět, že rozdíli dosahují maximálně desítek sekund pro větší instance na menším počtu procesorů. Částečný vliv na hodnoty má samozřejmě také momentální zatížení svazku STAR a jeho komunikačních sítí v době měření. V procentuálních hodnotách má nějvětší rozdíl hodnotu 5\% pro zadání č.1 na 2 jádrech, pro menší instance a také větší počet procesorů jsou ale rozdíly menší a tak můžeme říci, že rozdílná rychlost přenosu mezi uzly nemá na celkovou dobu výpočtu zásadní vliv.

\section{Závěr}

Úspěšně se nám podařilo implementovat sekvenční i paralelní algoritmu pro řešení zadaného problému.
Provedli jsme měření času potřebného pro výpočet  úloh netriviální složitosti sekvenčně a paralelně na 1-24 jádrech procesoru.
Výsledky měření ukazují, že algoritmus vykazuje na více procesorech superlineární zrychlení, a to především proto, že současným 
výpočtem ve více větvích stromu dokážeme dříve nalézt \uv{neideální} řešení a tedy snižovat horní mez, která je globálně sdílena.

Semestrální práce pro nás byla velkým přínosem, hlavně co se týká seznámení s MPI a komunikací v paralelním prostředí. 

\end{document}
